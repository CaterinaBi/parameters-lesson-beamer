%=+=+=+=+=+=+=+=+=+=+=+=+=+=+=+=+=+=+=+=+=+=+=+=+=+=+=+=+=+=+=+=+=+=+=+=+=+=+=+=
% Author: Mark H. Olson
% Website: https://mholson.com
% Github: https://github.com/mholson
%
% Updated: 2022-11-05 > Version 3.14159
% - adding support for minted nord color style from pygments
% - deprecating default support for listings package
% - initial Overleaf release
% Updated: 2022-07-08 > Version 3.1415
% - porting for ease of use with overleaf.com
% - changed code printing behavior by adding default support of listings pkg
% while leaving support for custom minted pkg using codeminted documentclass
% option
% Updated: 2022-04-28 > Version 3.141
% - fixed lists by removing enumitem package
% Updated: 2022-04-13 > Version 3.14
% - added optional bibliography
% - added template file
% Updated: 2022-04-13 Version 3.1
% Created: 2015-07-31 Version 3.0
%
%=+=+=+=+=+=+=+=+=+=+=+=+=+=+=+=+=+=+=+=+=+=+=+=+=+=+=+=+=+=+=+=+=+=+=+=+=+=+=+=

%=-=-=-=-=-=-=-=-=-=-=-=-=-=-=-=-=-=-=-=-=-=-=-=-=-=-=-=-=-=-=-=-=-=-=-=-=-=-=-=
% PREAMBLE :: sthlmNordLightDemo.tex
%=-=-=-=-=-=-=-=-=-=-=-=-=-=-=-=-=-=-=-=-=-=-=-=-=-=-=-=-=-=-=-=-=-=-=-=-=-=-=-=
%
% > > >	The following beamer class options are available
%		aspectratio=169		Change aspect ratio to 16:9
%		bibref				Include bibliography
%		sectionpages		Show section pages
%		codeminted			use minted pkg for code printing instead of listings
%							(requires additional setup & Python installed)
%		codemintedoverleaf	use minted pkg with color style support for Overleaf
% 		font sizes 			{8, 9, 10, 11, 12, 14, 17, 20} 11 Default
%
% > > > The following sthlmnord package options are available
%		mode				= dark (default)
%							= light
%=-=-=-=-=-=-=-=-=-=-=-=-=-=-=-=-=-=-=-=-=-=-=-=-=-=-=-=-=-=-=-=-=-=-=-=-=-=-=-=

\documentclass[aspectratio=169, sectionpages, codemintedoverleaf, bibref]{beamer}
% > > > Bibliography File
\newcommand{\bibfilename}{mhoRef.bib}
% > > > Choose Theme
\usetheme[mode=light]{sthlmnord}
%\usetheme{sthlmnord}

% > > > Generate some Lorem Ipsum placeholder text for the demo.
\usepackage{lipsum}

% > > >	Image File Paths
% 		Here you can add one or more paths to where your images are being
%		stored.  This will allow you to include only the image file
%		name when placing it into your document.
%\graphicspath{{path1},{path2},{path3}}
\graphicspath{{./assets/}}
% > > >	Optional use of using subfiles to make content more modular
\usepackage{subfiles}

% > > > Document Information
\title{Projections fonctionnelles et Paramètres}
\subtitle{Le cas de l'évolution de l'interrogation partielle en français.}
\newcommand{\titleAuthor}{Auteurs}
\author{Caterina Bonan \& Giuseppe Samo}
\newcommand{\titleInstitute}{Affiliations}
\institute{University of Cambridge; Université de Genève \& BLCU.}
\newcommand{\titleMiscI}{Cours}
\newcommand{\descMiscI}{Lectures en syntaxe contemporaine (Ur Shlonsky)}
\newcommand{\titleMiscII}{Date}
\newcommand{\descMiscII}{18 Décembre 2022}
%\date{\today}
\titlegraphic{SNF_logo.png}

% > > > pdf customizations via hyperref (pkg installed by beamer)
\hypersetup{
%colorlinks=true,
% You might want to disable color links for you final draft
% AND for colors to work properly where links are included.
colorlinks=false,
linkcolor={nordNine},
citecolor={nordNine},
urlcolor={nordNine}
}

%=-=-=-=-=-=-=-=-=-=-=-=-=-=-=-=-=-=-=-=-=-=-=-=-=-=-=-=-=-=-=-=-=-=-=-=-=-=-=-=
%
%    DOCUMENT BEGINS HERE 
%
%=-=-=-=-=-=-=-=-=-=-=-=-=-=-=-=-=-=-=-=-=-=-=-=-=-=-=-=-=-=-=-=-=-=-=-=-=-=-=-=
\begin{document}

%=-=-=-=-=-=-=-=-=-=-=-=-=-=-=-=-=-=-=-=-=-=-=-=-=-=-=-=-=-=-=-=-=-=-=-=-=-=-=-=
%   TITLE START   -=-=-=-=-=-=-=-=-=-=-=-=-=-=-=-=-=-=-=-=-=-=-=-=-=-=-=-=-=-=-=
\titlepage
%   TITLE END   --==-=-=-=-=-=-=-=-=-=-=-=-=-=-=-=-=-=-=-=-=-=-=-=-=-=-=-=-=-=-=
%=-=-=-=-=-=-=-=-=-=-=-=-=-=-=-=-=-=-=-=-=-=-=-=-=-=-=-=-=-=-=-=-=-=-=-=-=-=-=-=

\begin{frame}
    \frametitle{Acknowledgements}
    \begin{center}
        Le travail de Caterina Bonan a été supporté par le Fond National Suisse, bourse Postdoc.mobility numéro 202764, projet 'Different types of focus in diachrony: formal and typological variations'.
    \end{center}
\end{frame}

%=-=-=-=-=-=-=-=-=-=-=-=-=-=-=-=-=-=-=-=-=-=-=-=-=-=-=-=-=-=-=-=-=-=-=-=-=-=-=-=
%   TABLE OF CONTENTS START   -=-=-=-=-=-=-=-=-=-=-=-=-=-=-=-=-=-=-=-=-=-=-=-=-=
\begin{frame}
	\frametitle{Table of contents}
	% > > > For longer presentations use \tableofcontents[hideallsubsections] option
	%		It is also possible to manually control the entries of the table of 
	% 		contents by sections.
	%\tableofcontents[sections={1-10}]
	%\framebreak
	%\tableofcontents[sections={11-15}]
	\tableofcontents
\end{frame}
%   TABLE OF CONTENTS END   -=-=-=-=-=-=-=-=-=-=-=-=-=-=-=-=-=-=-=-=-=-=-=-=-=-=
%=-=-=-=-=-=-=-=-=-=-=-=-=-=-=-=-=-=-=-=-=-=-=-=-=-=-=-=-=-=-=-=-=-=-=-=-=-=-=-=

\section{Introduction}

\subfile{slides/introduction.tex}

\section{Les paramètres de Rizzi}

\subfile{slides/théorie.tex}
\subfile{slides/variation.tex}
\subfile{slides/complications.tex}

\section{Nos hypothèses}
\subfile{slides/hypothèses.tex}

\section{L'interrogation partielle en français}

\subfile{slides/types.tex}
\subfile{slides/explication.tex}

\section{Notre étude}

\subfile{slides/métodologie.tex}
\subfile{slides/résultats.tex}
\subfile{slides/discussion.tex}

\section{Conclusions}
\subfile{slides/conclusions.tex}

\section{References}

\begin{frame}[allowframebreaks]{References}
	\printbibliography[title={References}]%
\end{frame}

\end{document}
%=+=+=+=+=+=+=+=+=+=+=+=+=+=+=+=+=+=+=+=+=+=+=+=+=+=+=+=+=+=+=+=+=+=+=+=+=+=+=+=
% END OF FILE
%=+=+=+=+=+=+=+=+=+=+=+=+=+=+=+=+=+=+=+=+=+=+=+=+=+=+=+=+=+=+=+=+=+=+=+=+=+=+=+=
