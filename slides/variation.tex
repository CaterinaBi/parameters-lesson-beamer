\documentclass[lesson_slides]{subfiles}
\usepackage{natbib}
\usepackage{graphicx}
% \graphicspath{ {./images/} }
\usepackage{enumerate}
\usepackage{pifont} % for ding
\usepackage{float} % keeps tables in the exact position they occupy in the code
\usepackage{gb4e} % leave last

\begin{document}
%%=-=-=-=-=-=-=-=-=-=-=-=-=-=-=-=-=-=-=-=-=-=-=-=-=-=-=-=-=-=-=-=-=-=-=-=-=-=-=-=
%   FRAME START   -=-=-=-=-=-=-=-=-=-=-=-=-=-=-=-=-=-=-=-=-=-=-=-=-=-=-=-=-=-=-=
\begin{frame}[c]{Variations in the functional spine}

    \transboxin<1>
    \transglitter<2>
    \transwipe<3>
    \noindent Cartography of syntactic structures (\citealt{cinquerizzi2010,rizzicinque2016}, a.o.): \pause
    \begin{enumerate}
        \item the functional spine of human language is universal; \pause
        \item the functional spine comprises of numerous rigidly ordered functional projections; \pause
        \item \textbf{nonetheless}, languages vary to the extent in which they: \pause
            \begin{itemize}
                \item[\ding{227}] activate the functional heads of the spine; \pause
                \item[\ding{227}] realise these projections using different strategies. 
            \end{itemize}
    \end{enumerate}

\end{frame}
%   FRAME END   --==-=-=-=-=-=-=-=-=-=-=-=-=-=-=-=-=-=-=-=-=-=-=-=-=-=-=-=-=-=-=
%=-=-=-=-=-=-=-=-=-=-=-=-=-=-=-=-=-=-=-=-=-=-=-=-=-=-=-=-=-=-=-=-=-=-=-=-=-=-=-=
%   FRAME START   -=-=-=-=-=-=-=-=-=-=-=-=-=-=-=-=-=-=-=-=-=-=-=-=-=-=-=-=-=-=-=
    \begin{frame}[c]{FocusP}
    
    \textbf{\textsc{the left periphery}} (Rizzi 1997)
    
    [\textsubscript{ForceP} [ Force\textsuperscript{0} [\textsubscript{TopP*} [ Top\textsuperscript{0} [\textsubscript{FocusP} [ Focus\textsuperscript{0} [\textsubscript{TopP*} [ Top\textsuperscript{0} [\textsubscript{FinP} [ Fin\textsuperscript{0} [\textsubscript{IP} [ I\textsuperscript{0} \dots ]]]]]]]]]]]]
      
    \end{frame}
%   FRAME END   --==-=-=-=-=-=-=-=-=-=-=-=-=-=-=-=-=-=-=-=-=-=-=-=-=-=-=-=-=-=-=
%   FRAME START   -=-=-=-=-=-=-=-=-=-=-=-=-=-=-=-=-=-=-=-=-=-=-=-=-=-=-=-=-=-=-=
\begin{frame}[c]{FocusP}

\textbf{\textsc{the left periphery}} (Rizzi 1997)

[\textsubscript{ForceP} [ Force\textsuperscript{0} [\textsubscript{TopP*} [ Top\textsuperscript{0} [\textsubscript{\hl{FocusP}} [ Focus\textsuperscript{0} [\textsubscript{TopP*} [ Top\textsuperscript{0} [\textsubscript{FinP} [ Fin\textsuperscript{0} [\textsubscript{IP} [ I\textsuperscript{0} \dots ]]]]]]]]]]]]


  
\end{frame}
%   FRAME END   --==-=-=-=-=-=-=-=-=-=-=-=-=-=-=-=-=-=-=-=-=-=-=-=-=-=-=-=-=-=-=
%   FRAME START   -=-=-=-=-=-=-=-=-=-=-=-=-=-=-=-=-=-=-=-=-=-=-=-=-=-=-=-=-=-=-=
\begin{frame}[c]{FocusP: attracted constituents}

    \transboxin<1>
    \transglitter<2>
    \transwipe<3>
    \textbf{\textsc{what does focusp attract?}} \pause

    \begin{itemize}
        \item [\ding{227}] Wh-elements (wh-fronting) \pause

        \begin{exe}
        \ex Standard Italian
            \gll Chi hai visto?\\
                who have\textsubscript{2PS} seen\\
            \glt \vspace*{-2mm} ‘Who did you see?’
        \label{ita}
        \end{exe} \pause
    
        [\textsubscript{ForceP} [ Force\textsuperscript{0} [\textsubscript{TopP*} [ Top\textsuperscript{0} [\textsubscript{FocusP} \textsc{\hl{chi}} [ \dots hai visto ]]]]]]]]]]]] ?
        
    \end{itemize}
    
  
\end{frame}
%   FRAME END   --==-=-=-=-=-=-=-=-=-=-=-=-=-=-=-=-=-=-=-=-=-=-=-=-=-=-=-=-=-=-=
%   FRAME START   -=-=-=-=-=-=-=-=-=-=-=-=-=-=-=-=-=-=-=-=-=-=-=-=-=-=-=-=-=-=-=
\begin{frame}[c]{FocusP: attracted constituents}

    \transboxin<1>
    \transglitter<2>
    \transwipe<3>
    \textbf{\textsc{what does focusp attract?}} \pause

    \begin{itemize}
        \item [\ding{227}] Focused constituents (focus fronting) \pause

        \begin{exe}
            \ex Standard Italian
                \gll Gianni ho visto, non Marco!\\
                    John have\textsubscript{1PS} seen, not Marco\\
                \glt \vspace*{-2mm} ‘I saw John, not Marco’
            \label{ita}
        \end{exe} \pause
    
        [\textsubscript{ForceP} [ Force\textsuperscript{0} [\textsubscript{TopP*} [ Top\textsuperscript{0} [\textsubscript{FocusP} \textsc{\hl{gianni}} [ \dots ho visto ]]]]]]]]]]]] !
        
    \end{itemize}
    
  
\end{frame}
%   FRAME END   --==-=-=-=-=-=-=-=-=-=-=-=-=-=-=-=-=-=-=-=-=-=-=-=-=-=-=-=-=-=-=
%=-=-=-=-=-=-=-=-=-=-=-=-=-=-=-=-=-=-=-=-=-=-=-=-=-=-=-=-=-=-=-=-=-=-=-=-=-=-=-=
%   FRAME START   -=-=-=-=-=-=-=-=-=-=-=-=-=-=-=-=-=-=-=-=-=-=-=-=-=-=-=-=-=-=-=
\begin{frame}[c]{Variations in the realisation of FocusP}

    \transboxin<1>
    \transglitter<2>
    \transwipe<3>
    \noindent FocusP (HLP) is not realised/exploited in the same way by all languages. \citeauthor{samo2019cartography} (\citealt{samo2019cartography}). \pause
    
    Focus\textsuperscript{0} triggers movement of an XP that bears a relevant focus feature and: \pause
        \begin{itemize}
            \item[\ding{227}] in languages such as Gungbe this head is phonetically realised (\citealt{aboh2004morphosyntax}); \pause
        
            \item[\ding{227}] in languages like Italian, the head is silent (\citealt{rizzi1997fine} and related); \pause
                
            \item[\ding{227}] in V2 languages, the head is activated by moving an already merged head, as in German.
        \end{itemize}
  
\end{frame}
%   FRAME END   --==-=-=-=-=-=-=-=-=-=-=-=-=-=-=-=-=-=-=-=-=-=-=-=-=-=-=-=-=-=-=
%=-=-=-=-=-=-=-=-=-=-=-=-=-=-=-=-=-=-=-=-=-=-=-=-=-=-=-=-=-=-=-=-=-=-=-=-=-=-=-=
%   FRAME START   -=-=-=-=-=-=-=-=-=-=-=-=-=-=-=-=-=-=-=-=-=-=-=-=-=-=-=-=-=-=-=
\begin{frame}[c]{FocusP: Gungbe}

    \begin{exe}
        \ex Gungbe (adapted from Aboh 2007: 85(9c))
            \gll [\textsubscript{FocusP}  	\textsc{kofi}\textsubscript{i}   [\textsubscript{Focus^0} 	wè 			[	ùn   		yró		\_\_\_\textsubscript{i}		]]]!\\
                {} Kofi        {} 			foc    {}    	1sg   	call {} {}\\
            \vspace{-3mm}
            \glt ‘I called \textsc{kofi} (as opposed to, for example, Enoch)’
    \label{gungbe}
    \end{exe}
  
\end{frame}
%   FRAME END   --==-=-=-=-=-=-=-=-=-=-=-=-=-=-=-=-=-=-=-=-=-=-=-=-=-=-=-=-=-=-=
%=-=-=-=-=-=-=-=-=-=-=-=-=-=-=-=-=-=-=-=-=-=-=-=-=-=-=-=-=-=-=-=-=-=-=-=-=-=-=-=
%   FRAME START   -=-=-=-=-=-=-=-=-=-=-=-=-=-=-=-=-=-=-=-=-=-=-=-=-=-=-=-=-=-=-=
\begin{frame}[c]{FocusP: Standard Italian}

    \begin{exe}
            \ex Italian (adapted from \citealt{samo2019cartography}: 146 (8))
                \gll [\textsubscript{FocusP} 	\textsc{il} \textsc{libro}\textsubscript{i}	    [\textsubscript{Focus^0} 	$\emptyset$ 		[	Gianni 		ha 		letto 	\_\_\_\textsubscript{i}	]]]!\\
                {} the book				{}		foc		{}	Gianni 		has 	read 	\_\_\_\\
                \vspace{-3mm}
                \glt ‘Gianni read \textsc{the} \textsc{book} (as opposed to, for example, the article)’	
                \label{ita}
        \end{exe}
  
\end{frame}
%   FRAME END   --==-=-=-=-=-=-=-=-=-=-=-=-=-=-=-=-=-=-=-=-=-=-=-=-=-=-=-=-=-=-=
%=-=-=-=-=-=-=-=-=-=-=-=-=-=-=-=-=-=-=-=-=-=-=-=-=-=-=-=-=-=-=-=-=-=-=-=-=-=-=-=
%   FRAME START   -=-=-=-=-=-=-=-=-=-=-=-=-=-=-=-=-=-=-=-=-=-=-=-=-=-=-=-=-=-=-=
\begin{frame}[c]{FocusP: German}

        \begin{exe}
            \ex	German (adapted from \citealt{samo2019cartography}: 146 (8))
                \gll [\textsubscript{SpecFoc} 	\textsc{dieses} 	\textsc{fresko} 	[\textsubscript{Focus^0} 	malte				[	Giotto ]]]\\
                {} this 			fresco			{}		painted.3sg	{}	Giotto \\
                \vspace*{-3mm}
                \glt ‘Giotto painted \textsc{this} \textsc{fresco} (as opposed to, for example, the one over there)’
                \label{ger}
        \end{exe}
  
\end{frame}
%   FRAME END   --==-=-=-=-=-=-=-=-=-=-=-=-=-=-=-=-=-=-=-=-=-=-=-=-=-=-=-=-=-=-=
%=-=-=-=-=-
%   FRAME START   -=-=-=-=-=-=-=-=-=-=-=-=-=-=-=-=-=-=-=-=-=-=-=-=-=-=-=-=-=-=-=
\begin{frame}[c]{Variations in the realisation of FocusP (ii)}

    \transboxin<1>
    \transglitter<2>
    \transwipe<3>
    \noindent \textbf{\textsc{variations in the activation of focusp}} \pause
    \begin{itemize}
        \item[\ding{227}] Gungbe merges FocusP and spells out Focus^0; \pause 
        \item[\ding{227}] Italian merges FocusP but \textbf{does not} spell out Focus^0; \pause
        \item[\ding{227}] German requires \textbf{both} head movement and phrasal movement. 
    \end{itemize}
  
\end{frame}
%   FRAME END   --==-=-=-=-=-=-=-=-=-=-=-=-=-=-=-=-=-=-=-=-=-=-=-=-=-=-=-=-=-=-=
%=-=-=-=-=-
%   FRAME START   -=-=-=-=-=-=-=-=-=-=-=-=-=-=-=-=-=-=-=-=-=-=-=-=-=-=-=-=-=-=-=
\begin{frame}[c]{Variations in the realisation of FocusP (iii)}

    \noindent \textbf{\textsc{variations in the activation of focusp}} 
    \begin{table}[H]
    \centering
        \begin{tabular}{|l|r|r|r|r|r|r|}
        \hline
         & Merge & Spell Out & Search & IM & Search\textsubscript{lex} & IM\textsubscript{lex} \\
        \hline
        Italian & 1 & 0 & 1 & 1 & 0 & 0 \\
        \hline
        Gungbe & 1 & 1 & 1 & 1 & 0 & 0 \\
        \hline
        German & 1 & 0 & 1 & 1 & 1 & 1 \\
        \hline
        \end{tabular}
    \caption{\label{tab:samp}Language variability in activating FocusP (\citealt{samo2019cartography}: 147 (10)).}
    \end{table}

\end{frame}
%   FRAME END   --==-=-=-=-=-=-=-=-=-=-=-=-=-=-=-=-=-=-=-=-=-=-=-=-=-=-=-=-=-=-=
%=-=-=-=-=-
\begin{frame}{Summing up: Why we love this framework}

    \transboxin<1>
    \transglitter<2>
    %\transwipe<3>
    \begin{itemize}
        \item[\ding{227}] the variability of syntactic strategies adopted by different languages thus stems from different combinations of the syntactic operations of Merge, Move and Spell Out; \pause
        \item[\ding{227}] the factorial combinations of the boolean operators result in fine cross-linguistic analyses of typological variations. 
    \end{itemize}
    
\end{frame}
\end{document}