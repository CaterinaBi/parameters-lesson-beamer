\documentclass[lesson_slides]{subfiles}
%\usepackage{natbib}
%\bibliographystyle{plainnat}
\newcommand{\bibfilename}{mhoRef.bib}
\usepackage{graphicx}
% \graphicspath{ {./images/} }
\usepackage{enumerate}
\usepackage{pifont} % for ding
\usepackage{float} % keeps tables in the exact position they occupy in the code
\usepackage{xcolor} % text colour
\usepackage{gb4e} % leave last

\begin{document}
%%=-=-=-=-=-=-=-=-=-=-=-=-=-=-=-=-=-=-=-=-=-=-=-=-=-=-=-=-=-=-=-=-=-=-=-=-=-=-=-=
%   FRAME START   -=-=-=-=-=-=-=-=-=-=-=-=-=-=-=-=-=-=-=-=-=-=-=-=-=-=-=-=-=-=-=
\begin{frame}[c]{What?}

    \transboxin<1>
    \transglitter<2>
    \transwipe<3>
    \begin{itemize}
        \item[\ding{227}] Parameters; \pause $\longrightarrow$ \cite{rizzi2017} \pause
        \item[\ding{227}] Parametrization of functional projections related to 'focus' (FocusP, wh-elements); \pause
        \item[\ding{227}] synchronic/diachronic point of view.
    \end{itemize}
    
\end{frame}
%   FRAME END   --==-=-=-=-=-=-=-=-=-=-=-=-=-=-=-=-=-=-=-=-=-=-=-=-=-=-=-=-=-=-=
%=-=-=-=-=-=-=-=-=-=-=-=-=-=-=-=-=-=-=-=-=-=-=-=-=-=-=-=-=-=-=-=-=-=-=-=-=-=-=-=
%%=-=-=-=-=-=-=-=-=-=-=-=-=-=-=-=-=-=-=-=-=-=-=-=-=-=-=-=-=-=-=-=-=-=-=-=-=-=-=-=
%   FRAME START   -=-=-=-=-=-=-=-=-=-=-=-=-=-=-=-=-=-=-=-=-=-=-=-=-=-=-=-=-=-=-=
\begin{frame}[c]{Pourquoi?}
  Nous les croyons extrêmement puissants pour:
\end{frame}
%   FRAME END   --==-=-=-=-=-=-=-=-=-=-=-=-=-=-=-=-=-=-=-=-=-=-=-=-=-=-=-=-=-=-=
%=-=-=-=-=-=-=-=-=-=-=-=-=-=-=-=-=-=-=-=-=-=-=-=-=-=-=-=-=-=-=-=-=-=-=-=-=-=-=-=
%   FRAME START   -=-=-=-=-=-=-=-=-=-=-=-=-=-=-=-=-=-=-=-=-=-=-=-=-=-=-=-=-=-=-=
\begin{frame}[c]{Comment?}
  Nous utiliserons le français pour vous le prouver.
\end{frame}
%   FRAME END   --==-=-=-=-=-=-=-=-=-=-=-=-=-=-=-=-=-=-=-=-=-=-=-=-=-=-=-=-=-=-=
%=-=-=-=-=-=-=-=-=-=-=-=-=-=-=-=-=-=-=-=-=-=-=-=-=-=-=-=-=-=-=-=-=-=-=-=-=-=-=-=
%   FRAME START   -=-=-=-=-=-=-=-=-=-=-=-=-=-=-=-=-=-=-=-=-=-=-=-=-=-=-=-=-=-=-=
\begin{frame}[c]{Qu'est-ce que cela nous dit?}
  Les langues évoluenten direction du manque de mouvement.
  Mais aussi...
\end{frame}
%   FRAME END   --==-=-=-=-=-=-=-=-=-=-=-=-=-=-=-=-=-=-=-=-=-=-=-=-=-=-=-=-=-=-=
%=-=-=-=-=-=-=-=-=-=-=-=-=-=-=-=-=-=-=-=-=-=-=-=-=-=-=-=-=-=-=-=-=-=-=-=-=-=-=-=
\end{document}