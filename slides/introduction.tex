\documentclass[lesson_slides]{subfiles}
%\usepackage{natbib}
%\bibliographystyle{plainnat}
\newcommand{\bibfilename}{mhoRef.bib}
\usepackage{graphicx}
% \graphicspath{ {./images/} }
\usepackage{enumerate}
\usepackage{pifont} % for ding
\usepackage{float} % keeps tables in the exact position they occupy in the code
\usepackage{xcolor} % text colour
\usepackage{gb4e} % leave last

\begin{document}
%%=-=-=-=-=-=-=-=-=-=-=-=-=-=-=-=-=-=-=-=-=-=-=-=-=-=-=-=-=-=-=-=-=-=-=-=-=-=-=-=
%   FRAME START   -=-=-=-=-=-=-=-=-=-=-=-=-=-=-=-=-=-=-=-=-=-=-=-=-=-=-=-=-=-=-=
\begin{frame}[c]{What?}

    \transboxin<1>
    \transglitter<2>
    \transwipe<3>
    \begin{itemize}
        \item[\ding{227}] Parameters; \pause $\longrightarrow$ \cite{rizzi2017} \pause
        \item[\ding{227}] Parametrization of functional projections related to 'focus';\\ \pause
        $\longrightarrow$ (FocusP, wh-elements); \pause
        \item[\ding{227}] synchronic/diachronic point of view.
    \end{itemize}
    
\end{frame}
%   FRAME END   --==-=-=-=-=-=-=-=-=-=-=-=-=-=-=-=-=-=-=-=-=-=-=-=-=-=-=-=-=-=-=
%=-=-=-=-=-=-=-=-=-=-=-=-=-=-=-=-=-=-=-=-=-=-=-=-=-=-=-=-=-=-=-=-=-=-=-=-=-=-=-=
%%=-=-=-=-=-=-=-=-=-=-=-=-=-=-=-=-=-=-=-=-=-=-=-=-=-=-=-=-=-=-=-=-=-=-=-=-=-=-=-=
%   FRAME START   -=-=-=-=-=-=-=-=-=-=-=-=-=-=-=-=-=-=-=-=-=-=-=-=-=-=-=-=-=-=-=
\begin{frame}[c]{Why?}

    \transboxin<1>
    \transglitter<2>
    \transwipe<3>
    Rizzi's (2017) understanding of Parameters provides a straightforward and powerful way to:
    \begin{itemize}
        \item[\ding{227}] understand the underlying differences among functional projections clause-internally; \pause 
        \item[\ding{227}] understand the differences in the ways a functional projection is realised cross-linguistically;
        \item[\ding{227}] understand change.
    \end{itemize}
  
\end{frame}
%   FRAME END   --==-=-=-=-=-=-=-=-=-=-=-=-=-=-=-=-=-=-=-=-=-=-=-=-=-=-=-=-=-=-=
%=-=-=-=-=-=-=-=-=-=-=-=-=-=-=-=-=-=-=-=-=-=-=-=-=-=-=-=-=-=-=-=-=-=-=-=-=-=-=-=
%   FRAME START   -=-=-=-=-=-=-=-=-=-=-=-=-=-=-=-=-=-=-=-=-=-=-=-=-=-=-=-=-=-=-=
\begin{frame}[c]{How?}

    \transboxin<1>
    \transglitter<2>
    \transwipe<3>
    \begin{itemize}
        \item[\ding{227}] theory (\cite{rizzi1997fine}) and applications to FocusP (\cite{samo2019cartography}); \pause 
        \item[\ding{227}] study on French wh-interrogatives \pause (1870-2014).
    \end{itemize}
\end{frame}
%   FRAME END   --==-=-=-=-=-=-=-=-=-=-=-=-=-=-=-=-=-=-=-=-=-=-=-=-=-=-=-=-=-=-=
%   FRAME START   -=-=-=-=-=-=-=-=-=-=-=-=-=-=-=-=-=-=-=-=-=-=-=-=-=-=-=-=-=-=-=
\begin{frame}[c]{Why French?}

    \transboxin<1>
    \transglitter<2>
    \transwipe<3>
    \textbf{\textsc{why french?}} \pause
    \begin{itemize}
        \item[\ding{227}] ex situ/in situ alternation (Quand l'as-tu vu? vs Tu l'as vu quand?); \pause 
        \item[\ding{227}] co-existence of numerous question-formation strategies (SV, VS, est-ce que, etc.) \pause
        \item[\ding{227}] exceptional evolution in the distribution of these structures over the last decades.
    \end{itemize}
\end{frame}
%   FRAME END   --==-=-=-=-=-=-=-=-=-=-=-=-=-=-=-=-=-=-=-=-=-=-=-=-=-=-=-=-=-=-=
%=-=-=-=-=-=-=-=-=-=-=-=-=-=-=-=-=-=-=-=-=-=-=-=-=-=-=-=-=-=-=-=-=-=-=-=-=-=-=-=
%   FRAME START   -=-=-=-=-=-=-=-=-=-=-=-=-=-=-=-=-=-=-=-=-=-=-=-=-=-=-=-=-=-=-=
\begin{frame}[c]{What will we claim?}

    \transboxin<1>
    \transglitter<2>
    \transwipe<3>
    \begin{itemize}
        \item[\ding{227}] 
        languages have be known to evolve in the direction of loss of phrasal movement (\cite{dadan2019} and references therein) \pause $\longrightarrow$ multiple wh-fronting (Latin) \pause > single wh-fronting (Romance) \pause > ex situ/in situ alternation (French, non-standard Romance varieties) \pause > generalised in situ.\pause
        \item[\ding{227}] we use French to claim that languages also evolve in the direction of loss of \textbf{head} movement, and loss of pronounced functional heads (SpellOut).
    \end{itemize}
\end{frame}

\end{frame}
%   FRAME END   --==-=-=-=-=-=-=-=-=-=-=-=-=-=-=-=-=-=-=-=-=-=-=-=-=-=-=-=-=-=-=
%=-=-=-=-=-=-=-=-=-=-=-=-=-=-=-=-=-=-=-=-=-=-=-=-=-=-=-=-=-=-=-=-=-=-=-=-=-=-=-=
\end{document}