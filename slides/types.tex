\documentclass[lesson_slides]{subfiles}
%\usepackage{natbib}
\usepackage{graphicx}
% \graphicspath{ {./images/} }
\usepackage{enumerate}
\usepackage{pifont} % for ding
\usepackage{float} % keeps tables in the exact position they occupy in the code
\usepackage{xcolor} % text colour
\usepackage{gb4e} % leave last

\begin{document}
%%=-=-=-=-=-=-=-=-=-=-=-=-=-=-=-=-=-=-=-=-=-=-=-=-=-=-=-=-=-=-=-=-=-=-=-=-=-=-=-=
%   FRAME START   -=-=-=-=-=-=-=-=-=-=-=-=-=-=-=-=-=-=-=-=-=-=-=-=-=-=-=-=-=-=-=
\begin{frame}[c]{Question-formation strategies in French}

    \transboxin<1>
    \transglitter<2>
    \transwipe<3>
    \noindent \textbf{\textsc{wh-questions}} \pause
    \begin{itemize}
        \item[\ding{227}] in situ: “Tu as vu qui?”; \pause
        \item[\ding{227}] ex situ: “Qui tu as vu?”; \pause 
        \item[\ding{227}] ex situ: “Qui est-ce que tu as vu?”; \pause 
        \item[\ding{227}] ex situ: “Qui as-tu vu?”; \pause 
        \item[\ding{227}] cleft: “C’est qui que tu as vu?”.
    \end{itemize}
   
\end{frame}
%   FRAME END   --==-=-=-=-=-=-=-=-=-=-=-=-=-=-=-=-=-=-=-=-=-=-=-=-=-=-=-=-=-=-=
%   FRAME START   -=-=-=-=-=-=-=-=-=-=-=-=-=-=-=-=-=-=-=-=-=-=-=-=-=-=-=-=-=-=-=
\begin{frame}[c]{Question-formation strategies in French}

    \noindent \textbf{\textsc{wh-questions}}
    \begin{itemize}
        \item[\ding{227}] in situ: “Tu as vu \hl{qui}?”;
        \item[\ding{227}] ex situ: “Qui tu as vu?”;
        \item[\ding{227}] ex situ: “Qui est-ce que tu as vu?”; 
        \item[\ding{227}] ex situ: “Qui as-tu vu?”;
        \item[\ding{227}] cleft: “C’est qui que tu as vu?”.
    \end{itemize}
   
\end{frame}
%   FRAME END   --==-=-=-=-=-=-=-=-=-=-=-=-=-=-=-=-=-=-=-=-=-=-=-=-=-=-=-=-=-=-=
%   FRAME START   -=-=-=-=-=-=-=-=-=-=-=-=-=-=-=-=-=-=-=-=-=-=-=-=-=-=-=-=-=-=-=
\begin{frame}[c]{Question-formation strategies in French}

    \noindent \textbf{\textsc{wh-questions}}
    \begin{itemize}
        \item[\ding{227}] in situ: “Tu as vu \hl{qui}?”;
        \item[\ding{227}] ex situ: “\hl{Qui} tu as vu?”;
        \item[\ding{227}] ex situ: “\hl{Qui} est-ce que tu as vu?”; 
        \item[\ding{227}] ex situ: “\hl{Qui} as-tu vu?”;
        \item[\ding{227}] cleft: “C’est qui que tu as vu?”.
    \end{itemize}
   
\end{frame}
%   FRAME END   --==-=-=-=-=-=-=-=-=-=-=-=-=-=-=-=-=-=-=-=-=-=-=-=-=-=-=-=-=-=-=
%   FRAME START   -=-=-=-=-=-=-=-=-=-=-=-=-=-=-=-=-=-=-=-=-=-=-=-=-=-=-=-=-=-=-=
\begin{frame}[c]{Question-formation strategies in French}

    \noindent \textbf{\textsc{wh-questions}}
    \begin{itemize}
        \item[\ding{227}] in situ: “Tu as vu \hl{qui}?”;
        \item[\ding{227}] ex situ: “\hl{Qui}\textsubscript{i} tu as vu \_\_\_\textsubscript{i}?”;
        \item[\ding{227}] ex situ: “\hl{Qui}\textsubscript{i} est-ce que tu as vu \_\_\_\textsubscript{i}?”; 
        \item[\ding{227}] ex situ: “\hl{Qui}\textsubscript{i} as-tu vu \_\_\_\textsubscript{i}?”;
        \item[\ding{227}] cleft: “C’est qui que tu as vu?”.
    \end{itemize}
   
\end{frame}
%   FRAME END   --==-=-=-=-=-=-=-=-=-=-=-=-=-=-=-=-=-=-=-=-=-=-=-=-=-=-=-=-=-=-=
%   FRAME START   -=-=-=-=-=-=-=-=-=-=-=-=-=-=-=-=-=-=-=-=-=-=-=-=-=-=-=-=-=-=-=
\begin{frame}[c]{Question-formation strategies in French}

    \noindent \textbf{\textsc{wh-questions}}
    \begin{itemize}
        \item[\ding{227}] in situ: “Tu as vu \hl{qui}?”; $\longrightarrow$ SV
        \item[\ding{227}] ex situ: “\hl{Qui}\textsubscript{i} tu as vu \_\_\_\textsubscript{i}?”;
        \item[\ding{227}] ex situ: “\hl{Qui}\textsubscript{i} est-ce que tu as vu \_\_\_\textsubscript{i}?”; 
        \item[\ding{227}] ex situ: “\hl{Qui}\textsubscript{i} as-tu vu \_\_\_\textsubscript{i}?”;
        \item[\ding{227}] cleft: “C’est qui que tu as vu?”.
    \end{itemize}
   
\end{frame}
%   FRAME END   --==-=-=-=-=-=-=-=-=-=-=-=-=-=-=-=-=-=-=-=-=-=-=-=-=-=-=-=-=-=-=
%   FRAME START   -=-=-=-=-=-=-=-=-=-=-=-=-=-=-=-=-=-=-=-=-=-=-=-=-=-=-=-=-=-=-=
\begin{frame}[c]{Question-formation strategies in French}

    \noindent \textbf{\textsc{wh-questions}}
    \begin{itemize}
        \item[\ding{227}] in situ: “\textbf{Tu} \textbf{as} \textbf{vu} \hl{qui}?”; $\longrightarrow$ SV
        \item[\ding{227}] ex situ: “\hl{Qui}\textsubscript{i} tu as vu \_\_\_\textsubscript{i}?”;
        \item[\ding{227}] ex situ: “\hl{Qui}\textsubscript{i} est-ce que tu as vu \_\_\_\textsubscript{i}?”; 
        \item[\ding{227}] ex situ: “\hl{Qui}\textsubscript{i} as-tu vu \_\_\_\textsubscript{i}?”;
        \item[\ding{227}] cleft: “C’est qui que tu as vu?”.
    \end{itemize}
   
\end{frame}
%   FRAME END   --==-=-=-=-=-=-=-=-=-=-=-=-=-=-=-=-=-=-=-=-=-=-=-=-=-=-=-=-=-=-=
%   FRAME START   -=-=-=-=-=-=-=-=-=-=-=-=-=-=-=-=-=-=-=-=-=-=-=-=-=-=-=-=-=-=-=
\begin{frame}[c]{Question-formation strategies in French}

    \noindent \textbf{\textsc{wh-questions}}
    \begin{itemize}
        \item[\ding{227}] in situ: “\textbf{Tu} \textbf{as} \textbf{vu} \hl{qui}?”; $\longrightarrow$ SV
        \item[\ding{227}] ex situ: “\hl{Qui}\textsubscript{i} \textbf{tu} \textbf{as} \textbf{vu} \_\_\_\textsubscript{i}?”;
        \item[\ding{227}] ex situ: “\hl{Qui}\textsubscript{i} est-ce que tu as vu \_\_\_\textsubscript{i}?”; 
        \item[\ding{227}] ex situ: “\hl{Qui}\textsubscript{i} as-tu vu \_\_\_\textsubscript{i}?”;
        \item[\ding{227}] cleft: “C’est qui que tu as vu?”.
    \end{itemize}
   
\end{frame}
%   FRAME END   --==-=-=-=-=-=-=-=-=-=-=-=-=-=-=-=-=-=-=-=-=-=-=-=-=-=-=-=-=-=-=
%   FRAME START   -=-=-=-=-=-=-=-=-=-=-=-=-=-=-=-=-=-=-=-=-=-=-=-=-=-=-=-=-=-=-=
\begin{frame}[c]{Question-formation strategies in French}

    \noindent \textbf{\textsc{wh-questions}}
    \begin{itemize}
        \item[\ding{227}] in situ: “\textbf{Tu} \textbf{as} \textbf{vu} \hl{qui}?”; $\longrightarrow$ SV
        \item[\ding{227}] ex situ: “\hl{Qui}\textsubscript{i} \textbf{tu} \textbf{as} \textbf{vu} \_\_\_\textsubscript{i}?”;  $\longrightarrow$ SV
        \item[\ding{227}] ex situ: “\hl{Qui}\textsubscript{i} est-ce que tu as vu \_\_\_\textsubscript{i}?”; 
        \item[\ding{227}] ex situ: “\hl{Qui}\textsubscript{i} as-tu vu \_\_\_\textsubscript{i}?”;
        \item[\ding{227}] cleft: “C’est qui que tu as vu?”.
    \end{itemize}
   
\end{frame}
%   FRAME END   --==-=-=-=-=-=-=-=-=-=-=-=-=-=-=-=-=-=-=-=-=-=-=-=-=-=-=-=-=-=-=
%   FRAME START   -=-=-=-=-=-=-=-=-=-=-=-=-=-=-=-=-=-=-=-=-=-=-=-=-=-=-=-=-=-=-=
\begin{frame}[c]{Question-formation strategies in French}

    \noindent \textbf{\textsc{wh-questions}}
    \begin{itemize}
        \item[\ding{227}] in situ: “\textbf{Tu} \textbf{as} \textbf{vu} \hl{qui}?”; $\longrightarrow$ \textcolor{red}{SV}
        \item[\ding{227}] ex situ: “\hl{Qui}\textsubscript{i} \textbf{tu} \textbf{as} \textbf{vu} \_\_\_\textsubscript{i}?”;  $\longrightarrow$ \textcolor{red}{SV}
        \item[\ding{227}] ex situ: “\hl{Qui}\textsubscript{i} est-ce que tu as vu \_\_\_\textsubscript{i}?”; 
        \item[\ding{227}] ex situ: “\hl{Qui}\textsubscript{i} as-tu vu \_\_\_\textsubscript{i}?”;
        \item[\ding{227}] cleft: “C’est qui que tu as vu?”.
    \end{itemize}
   
\end{frame}
%   FRAME END   --==-=-=-=-=-=-=-=-=-=-=-=-=-=-=-=-=-=-=-=-=-=-=-=-=-=-=-=-=-=-=
%   FRAME START   -=-=-=-=-=-=-=-=-=-=-=-=-=-=-=-=-=-=-=-=-=-=-=-=-=-=-=-=-=-=-=
\begin{frame}[c]{Question-formation strategies in French}

    \noindent \textbf{\textsc{wh-questions}}
    \begin{itemize}
        \item[\ding{227}] in situ: “\textbf{Tu} \textbf{as} \textbf{vu} \hl{qui}?”; $\longrightarrow$ \textcolor{red}{SV}
        \item[\ding{227}] ex situ: “\hl{Qui}\textsubscript{i} \textbf{tu} \textbf{as} \textbf{vu} \_\_\_\textsubscript{i}?”;  $\longrightarrow$ \textcolor{red}{SV}
        \item[\ding{227}] \underline{ex situ}: “\hl{Qui}\textsubscript{i} est-ce que tu as vu \_\_\_\textsubscript{i}?”; 
        \item[\ding{227}] \underline{ex situ}: “\hl{Qui}\textsubscript{i} as-tu vu \_\_\_\textsubscript{i}?”;
        \item[\ding{227}] cleft: “C’est qui que tu as vu?”.
    \end{itemize}
   
\end{frame}
%   FRAME END   --==-=-=-=-=-=-=-=-=-=-=-=-=-=-=-=-=-=-=-=-=-=-=-=-=-=-=-=-=-=-=
%   FRAME START   -=-=-=-=-=-=-=-=-=-=-=-=-=-=-=-=-=-=-=-=-=-=-=-=-=-=-=-=-=-=-=
\begin{frame}[c]{Question-formation strategies in French}

    \noindent \textbf{\textsc{wh-questions}}
    \begin{itemize}
        \item[\ding{227}] in situ: “\textbf{Tu} \textbf{as} \textbf{vu} \hl{qui}?”; $\longrightarrow$ \textcolor{red}{SV}
        \item[\ding{227}] ex situ: “\hl{Qui}\textsubscript{i} \textbf{tu} \textbf{as} \textbf{vu} \_\_\_\textsubscript{i}?”;  $\longrightarrow$ \textcolor{red}{SV}
        \item[\ding{227}] \underline{ex situ}: “\hl{Qui}\textsubscript{i} \textbf{est-ce que} tu as vu \_\_\_\textsubscript{i}?”; $\longrightarrow$ \textcolor{blue}{SV}
        \item[\ding{227}] \underline{ex situ}: “\hl{Qui}\textsubscript{i} as-tu vu \_\_\_\textsubscript{i}?”;
        \item[\ding{227}] cleft: “C’est qui que tu as vu?”.
    \end{itemize}
   
\end{frame}
%   FRAME END   --==-=-=-=-=-=-=-=-=-=-=-=-=-=-=-=-=-=-=-=-=-=-=-=-=-=-=-=-=-=-=
%   FRAME START   -=-=-=-=-=-=-=-=-=-=-=-=-=-=-=-=-=-=-=-=-=-=-=-=-=-=-=-=-=-=-=
\begin{frame}[c]{Question-formation strategies in French}

    \noindent \textbf{\textsc{wh-questions}}
    \begin{itemize}
        \item[\ding{227}] in situ: “\textbf{Tu} \textbf{as} \textbf{vu} \hl{qui}?”; $\longrightarrow$ \textcolor{red}{SV}
        \item[\ding{227}] ex situ: “\hl{Qui}\textsubscript{i} \textbf{tu} \textbf{as} \textbf{vu} \_\_\_\textsubscript{i}?”;  $\longrightarrow$ \textcolor{red}{SV}
        \item[\ding{227}] \underline{ex situ}: “\hl{Qui}\textsubscript{i} \textbf{est-ce que} tu as vu \_\_\_\textsubscript{i}?”; $\longrightarrow$ \textcolor{blue}{SV}
        \item[\ding{227}] \underline{ex situ}: “\hl{Qui}\textsubscript{i} \textbf{as-tu} vu \_\_\_\textsubscript{i}?”;
        \item[\ding{227}] cleft: “C’est qui que tu as vu?”.
    \end{itemize}
   
\end{frame}
%   FRAME END   --==-=-=-=-=-=-=-=-=-=-=-=-=-=-=-=-=-=-=-=-=-=-=-=-=-=-=-=-=-=-=
%   FRAME START   -=-=-=-=-=-=-=-=-=-=-=-=-=-=-=-=-=-=-=-=-=-=-=-=-=-=-=-=-=-=-=
\begin{frame}[c]{Question-formation strategies in French}

    \noindent \textbf{\textsc{wh-questions}}
    \begin{itemize}
        \item[\ding{227}] in situ: “\textbf{Tu} \textbf{as} \textbf{vu} \hl{qui}?”; $\longrightarrow$ \textcolor{red}{SV}
        \item[\ding{227}] ex situ: “\hl{Qui}\textsubscript{i} \textbf{tu} \textbf{as} \textbf{vu} \_\_\_\textsubscript{i}?”;  $\longrightarrow$ \textcolor{red}{SV}
        \item[\ding{227}] \underline{ex situ}: “\hl{Qui}\textsubscript{i} \textbf{est-ce que} tu as vu \_\_\_\textsubscript{i}?”; $\longrightarrow$ \textcolor{blue}{SV}
        \item[\ding{227}] \underline{ex situ}: “\hl{Qui}\textsubscript{i} \textbf{as-tu} vu \_\_\_\textsubscript{i}?”; $\longrightarrow$ \textcolor{purple}{VS}
        \item[\ding{227}] cleft: “C’est qui que tu as vu?”.
    \end{itemize}
   
\end{frame}
%   FRAME END   --==-=-=-=-=-=-=-=-=-=-=-=-=-=-=-=-=-=-=-=-=-=-=-=-=-=-=-=-=-=-=
%   FRAME START   -=-=-=-=-=-=-=-=-=-=-=-=-=-=-=-=-=-=-=-=-=-=-=-=-=-=-=-=-=-=-=
\begin{frame}[c]{Question-formation strategies in French}

    \noindent \textbf{\textsc{wh-questions}}
    \begin{itemize}
        \item[\ding{227}] in situ: “\textbf{Tu} \textbf{as} \textbf{vu} \hl{qui}?”; $\longrightarrow$ \textcolor{red}{SV}
        \item[\ding{227}] ex situ: “\hl{Qui}\textsubscript{i} \textbf{tu} \textbf{as} \textbf{vu} \_\_\_\textsubscript{i}?”;  $\longrightarrow$ \textcolor{red}{SV}
        \item[\ding{227}] \underline{ex situ}: “\hl{Qui}\textsubscript{i} \textbf{est-ce que} tu as vu \_\_\_\textsubscript{i}?”; $\longrightarrow$ \textcolor{blue}{SV}
        \item[\ding{227}] \underline{ex situ}: “\hl{Qui}\textsubscript{i} \textbf{as-tu} vu \_\_\_\textsubscript{i}?”; $\longrightarrow$ \textcolor{purple}{VS} (subject-clitic inversion)
        \item[\ding{227}] cleft: “C’est qui que tu as vu?”.
    \end{itemize}
   
\end{frame}
%   FRAME END   --==-=-=-=-=-=-=-=-=-=-=-=-=-=-=-=-=-=-=-=-=-=-=-=-=-=-=-=-=-=-=
%   FRAME START   -=-=-=-=-=-=-=-=-=-=-=-=-=-=-=-=-=-=-=-=-=-=-=-=-=-=-=-=-=-=-=
\begin{frame}[c]{Question-formation strategies in French}

    \noindent \textbf{\textsc{wh-questions}}
    \begin{itemize}
        \item[\ding{227}] in situ: “\textbf{Tu} \textbf{as} \textbf{vu} \hl{qui}?”; $\longrightarrow$ \textcolor{red}{SV}
        \item[\ding{227}] ex situ: “\hl{Qui}\textsubscript{i} \textbf{tu} \textbf{as} \textbf{vu} \_\_\_\textsubscript{i}?”;  $\longrightarrow$ \textcolor{red}{SV}
        \item[\ding{227}] \underline{ex situ}: “\hl{Qui}\textsubscript{i} \textbf{est-ce que} tu as vu \_\_\_\textsubscript{i}?”; $\longrightarrow$ \textcolor{blue}{SV}
        \item[\ding{227}] \underline{ex situ}: “\hl{Qui}\textsubscript{i} \textbf{as-tu} vu \_\_\_\textsubscript{i}?”; $\longrightarrow$ \textcolor{purple}{VS} (subject-clitic inversion)
        \item[\ding{227}] (cleft: “C’est qui que tu as vu?”.)
    \end{itemize}
   
\end{frame}
%   FRAME END   --==-=-=-=-=-=-=-=-=-=-=-=-=-=-=-=-=-=-=-=-=-=-=-=-=-=-=-=-=-=-=
%   FRAME START   -=-=-=-=-=-=-=-=-=-=-=-=-=-=-=-=-=-=-=-=-=-=-=-=-=-=-=-=-=-=-=
\begin{frame}[c]{Question-formation strategies in French}

    \noindent \textbf{\textsc{wh-questions}}
    \begin{itemize}
        \item[\ding{227}] in situ: “\textbf{Tu} \textbf{as} \textbf{vu} \hl{qui}?”; $\longrightarrow$ \textcolor{red}{SV}
        \item[\ding{227}] ex situ: “\hl{Qui}\textsubscript{i} \textbf{tu} \textbf{as} \textbf{vu} \_\_\_\textsubscript{i}?”;  $\longrightarrow$ \textcolor{red}{SV}
        \item[\ding{227}] \underline{ex situ}: “\hl{Qui}\textsubscript{i} \textbf{est-ce que} tu as vu \_\_\_\textsubscript{i}?”; $\longrightarrow$ \textcolor{blue}{SV}
        \item[\ding{227}] \underline{ex situ}: “\hl{Qui}\textsubscript{i} \textbf{as-tu} vu \_\_\_\textsubscript{i}?”; $\longrightarrow$ \textcolor{purple}{VS} (subject-clitic inversion)
        \item[\ding{227}] (cleft: “C’est qui que tu as vu?”.) $\longrightarrow$ bi-clausal
    \end{itemize}
   
\end{frame}
%   FRAME END   --==-=-=-=-=-=-=-=-=-=-=-=-=-=-=-=-=-=-=-=-=-=-=-=-=-=-=-=-=-=-=
%   FRAME START   -=-=-=-=-=-=-=-=-=-=-=-=-=-=-=-=-=-=-=-=-=-=-=-=-=-=-=-=-=-=-=
\begin{frame}[c]{Question-formation strategies in French}

    \noindent \textbf{\textsc{wh-questions}}
    \begin{itemize}
        \item[\ding{227}] in situ: “\textbf{Tu} \textbf{as} \textbf{vu} \hl{qui}?”; $\longrightarrow$ \textcolor{red}{SV}
        \item[\ding{227}] ex situ: “\hl{Qui}\textsubscript{i} \textbf{tu} \textbf{as} \textbf{vu} \_\_\_\textsubscript{i}?”;  $\longrightarrow$ \textcolor{red}{SV}
        \item[\ding{227}] \underline{ex situ}: “\hl{Qui}\textsubscript{i} \textbf{est-ce que} tu as vu \_\_\_\textsubscript{i}?”; $\longrightarrow$ \textcolor{blue}{SV}
        \item[\ding{227}] \underline{ex situ}: “\hl{Qui}\textsubscript{i} \textbf{as-tu} vu \_\_\_\textsubscript{i}?”; $\longrightarrow$ \textcolor{purple}{VS} (subject-clitic inversion)
        \item[\ding{227}] (cleft: “C’est qui que tu as vu?”.) $\longrightarrow$ bi-clausal (Belletti 2009, Haegeman et al. 2012)
    \end{itemize}
   
\end{frame}
%   FRAME END   --==-=-=-=-=-=-=-=-=-=-=-=-=-=-=-=-=-=-=-=-=-=-=-=-=-=-=-=-=-=-=
%=-=-=-=-=-=-=-=-=-=-=-=-=-=-=-=-=-=-=-=-=-=-=-=-=-=-=-=-=-=-=-=-=-=-=-=-=-=-=-=
%=-=-=-=-=-=-=-=-=-=-=-=-=-=-=-=-=-=-=-=-=-=-=-=-=-=-=-=-=-=-=-=-=-=-=-=-=-=-=-=
\end{document}