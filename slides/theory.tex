\documentclass[lesson_slides]{subfiles}
\usepackage{natbib}
\usepackage{graphicx}
% \graphicspath{ {./images/} }
\usepackage{enumerate}
\usepackage{pifont} % for ding
\usepackage{float} % keeps tables in the exact position they occupy in the code
\usepackage{gb4e} % leave last

\begin{document}
%%=-=-=-=-=-=-=-=-=-=-=-=-=-=-=-=-=-=-=-=-=-=-=-=-=-=-=-=-=-=-=-=-=-=-=-=-=-=-=-=
%   FRAME START   -=-=-=-=-=-=-=-=-=-=-=-=-=-=-=-=-=-=-=-=-=-=-=-=-=-=-=-=-=-=-=
\begin{frame}[c]{What we know about how languages evolve}

    \transboxin<1>
        \transglitter<2>
        \transwipe<3>
        \noindent \textbf{\textsc{languages evolve}} \pause
        \begin{itemize}
            \item[\ding{227}] in the direction of no movement of phrases (IM$=$0); \pause
            \item[\ding{227}] through optional stages (IM$=$0/1).
        \end{itemize}


\end{frame}
%   FRAME END   --==-=-=-=-=-=-=-=-=-=-=-=-=-=-=-=-=-=-=-=-=-=-=-=-=-=-=-=-=-=-=
%=-=-=-=-=-=-=-=-=-=-=-=-=-=-=-=-=-=-=-=-=-=-=-=-=-=-=-=-=-=-=-=-=-=-=-=-=-=-=-=
%   FRAME START   -=-=-=-=-=-=-=-=-=-=-=-=-=-=-=-=-=-=-=-=-=-=-=-=-=-=-=-=-=-=-=
\begin{frame}[c]{IM$=$1 vs IM$=$0}

    \transboxin<1>
    \transglitter<2>
    \transwipe<3>
    \noindent \textbf{\textsc{low movement vs no movement: ifoc}} \pause

    \begin{exe}
    \ex Low focus in Italian \pause
        \begin{xlist}
        \ex Question: Who's arrived?
        \ex \gll Answer A: ?? \textsc{gianni} è arrivato\\
        {} {} {} John is arrived\\ \pause
        \ex \gll Answer B: È arrivato \textsc{gianni}\\
        {} {} is arrived John\\ 
        \glt \hspace{16mm} '\textsc{john} arrived' (Lit: 'Arrived John')
        \end{xlist}
    \end{exe}

\end{frame}
%   FRAME END   --==-=-=-=-=-=-=-=-=-=-=-=-=-=-=-=-=-=-=-=-=-=-=-=-=-=-=-=-=-=-=
%   FRAME START   -=-=-=-=-=-=-=-=-=-=-=-=-=-=-=-=-=-=-=-=-=-=-=-=-=-=-=-=-=-=-=
\begin{frame}[c]{IM$=$1 vs IM$=$0}

    \transboxin<1>
    \transglitter<2>
    \transwipe<3>
    \noindent \textbf{\textsc{low movement vs no movement: ifoc}}

    \begin{exe}
    \ex VS structure in Italian (ii) \pause
        \begin{xlist}
            \ex \gll * Ho dato \textsc{a} \textsc{gianni} la merendina\\
            {} have\textsubscript{1PS} given to John the snack\\  
            \ex \gll Ho dato la merendina \textsc{a} \textsc{gianni}\\
            have\textsubscript{1PS} given the snack to John\\
            \glt 'I gave the snack \textsc{to} \textsc{john}' \pause
            \ex Question: To whom did you give your snack?
        \end{xlist}
    \end{exe}

\end{frame}
%   FRAME END   --==-=-=-=-=-=-=-=-=-=-=-=-=-=-=-=-=-=-=-=-=-=-=-=-=-=-=-=-=-=-=
%   FRAME START   -=-=-=-=-=-=-=-=-=-=-=-=-=-=-=-=-=-=-=-=-=-=-=-=-=-=-=-=-=-=-=
\begin{frame}[c]{IM$=$1 vs IM$=$0}

    \transboxin<1>
    \transglitter<2>
    \transwipe<3>
    \noindent \textbf{\textsc{low movement vs no movement: ifoc}}

    \begin{exe}
        \ex VS structure in Trevisan \pause
            \begin{xlist}
                \ex Question: To whom did you give your snack? \pause
                \ex \gll ?? Ghe go dato a marenda \textsc{a} \textsc{giani}\\
                {} 3dat have\textsubscript{1PS} given the snack to John\\
                \ex \gll Ghe go dato \textsc{a} \textsc{giani} a marenda\\
                3dat have\textsubscript{1PS} given to John the snack\\
                \glt 'Lit: I gave \textsc{to} \textsc{john} my snack.'
            \end{xlist}
    \end{exe}

\end{frame}
%   FRAME END   --==-=-=-=-=-=-=-=-=-=-=-=-=-=-=-=-=-=-=-=-=-=-=-=-=-=-=-=-=-=-=
%   FRAME START   -=-=-=-=-=-=-=-=-=-=-=-=-=-=-=-=-=-=-=-=-=-=-=-=-=-=-=-=-=-=-=
\begin{frame}[c]{IM$=$1 vs IM$=$0}

\begin{table}[ht]
    \centering
    \begin{tabular}{|l|r|r|r|r|r|r|}
    \hline
     & Merge & Spell Out & Search & IM & Search\textsubscript{lex} & IM\textsubscript{lex} \\
    \hline
    Trevisan & 1 & 0 & 1 & 1 & 0 & 0\\
    \hline
    Italian & 1 & 0 & 1 & 0 & 0 & 0 \\
    \hline
    \end{tabular}
    \caption{\label{tab:samp2}Informational foci: Trevisan vs Italian.}
    \end{table}

\end{frame}
%   FRAME END   --==-=-=-=-=-=-=-=-=-=-=-=-=-=-=-=-=-=-=-=-=-=-=-=-=-=-=-=-=-=-=
%   FRAME START   -=-=-=-=-=-=-=-=-=-=-=-=-=-=-=-=-=-=-=-=-=-=-=-=-=-=-=-=-=-=-=
\begin{frame}[c]{IM$=$1 vs IM$=$0}

\begin{table}[ht]
    \centering
    \begin{tabular}{|l|r|r|r|r|r|r|}
    \hline
     & Merge & Spell Out & Search & IM & Search\textsubscript{lex} & IM\textsubscript{lex} \\
    \hline
    Trevisan & 1 & 0 & 1 & \hl{1} & 0 & 0\\
    \hline
    Italian & 1 & 0 & 1 & \hl{0} & 0 & 0 \\
    \hline
    \end{tabular}
    \caption{\label{tab:samp2}Informational foci: Trevisan vs Italian.}
    \end{table}

\end{frame}
%   FRAME END   --==-=-=-=-=-=-=-=-=-=-=-=-=-=-=-=-=-=-=-=-=-=-=-=-=-=-=-=-=-=-=
%   FRAME START   -=-=-=-=-=-=-=-=-=-=-=-=-=-=-=-=-=-=-=-=-=-=-=-=-=-=-=-=-=-=-=
\begin{frame}[c]{IM$=$1 to IM$=$0}

\textbf{The parametrisation of functional projections evolves in the direction of no movement} (IM=0 in \citeauthor{rizzi2017}’s \citeyear{rizzi2017} terms, see works on the diachrony of Chinese interrogatives \citealt{aldridge2010clause} or Japanese, \citealt{aldridge2009old}, but also Latin and the Romance languages, \citealt{roberts2003syntactic}, \citealt{dadan2019}, a.o.).

\begin{table}[ht]
    \centering
    \begin{tabular}{|l|r|r|r|r|r|r|}
    \hline
     & Merge & Spell Out & Search & IM & Search\textsubscript{lex} & IM\textsubscript{lex} \\
    \hline
    Old Japanese ($=$Trevisan) & 1 & 0 & 1 & 1 & 0 & 0\\
    \hline
    Contemporary Japanese ($=$Italian) & 1 & 0 & 1 & 0 & 0 & 0 \\
    \hline
    \end{tabular}
    \caption{\label{tab:samp2}Parametrisations of FocP in the evolution of Japanese.}
    \end{table}

\end{frame}
%   FRAME END   --==-=-=-=-=-=-=-=-=-=-=-=-=-=-=-=-=-=-=-=-=-=-=-=-=-=-=-=-=-=-=
%   FRAME START   -=-=-=-=-=-=-=-=-=-=-=-=-=-=-=-=-=-=-=-=-=-=-=-=-=-=-=-=-=-=-=
\begin{frame}[c]{IM$=$1 to IM$=$0}

\textbf{The parametrisation of functional projections evolves in the direction of no movement} (IM=0 in \citeauthor{rizzi2017}’s \citeyear{rizzi2017} terms, see works on the diachrony of Chinese interrogatives \citealt{aldridge2010clause} or Japanese, \citealt{aldridge2009old}, but also Latin and the Romance languages, \citealt{roberts2003syntactic}, \citealt{dadan2019}, a.o.).

\begin{table}[ht]
    \centering
    \begin{tabular}{|l|r|r|r|r|r|r|}
    \hline
     & Merge & Spell Out & Search & IM & Search\textsubscript{lex} & IM\textsubscript{lex} \\
    \hline
    Old Japanese ($=$Trevisan) & 1 & 0 & 1 & \hl{1} & 0 & 0\\
    \hline
    Contemporary Japanese ($=$Italian) & 1 & 0 & 1 & 0 & 0 & 0 \\
    \hline
    \end{tabular}
    \caption{\label{tab:samp2}Parametrisations of FocP in the evolution of Japanese.}
    \end{table}

\end{frame}
%   FRAME END   --==-=-=-=-=-=-=-=-=-=-=-=-=-=-=-=-=-=-=-=-=-=-=-=-=-=-=-=-=-=-=

%   FRAME START   -=-=-=-=-=-=-=-=-=-=-=-=-=-=-=-=-=-=-=-=-=-=-=-=-=-=-=-=-=-=-=
\begin{frame}[c]{IM$=$1 to IM$=$0}

\textbf{The parametrisation of functional projections evolves in the direction of no movement} (IM=0 in \citeauthor{rizzi2017}’s \citeyear{rizzi2017} terms, see works on the diachrony of Chinese interrogatives \citealt{aldridge2010clause} or Japanese, \citealt{aldridge2009old}, but also Latin and the Romance languages, \citealt{roberts2003syntactic}, \citealt{dadan2019}, a.o.).

\begin{table}[ht]
    \centering
    \begin{tabular}{|l|r|r|r|r|r|r|}
    \hline
     & Merge & Spell Out & Search & IM & Search\textsubscript{lex} & IM\textsubscript{lex} \\
    \hline
    Old Japanese ($=$Trevisan) & 1 & 0 & 1 & 1 & 0 & 0\\
    \hline
    Contemporary Japanese ($=$Italian) & 1 & 0 & 1 & \hl{0} & 0 & 0 \\
    \hline
    \end{tabular}
    \caption{\label{tab:samp2}Parametrisations of FocP in the evolution of Japanese.}
    \end{table}

\end{frame}
%   FRAME END   --==-=-=-=-=-=-=-=-=-=-=-=-=-=-=-=-=-=-=-=-=-=-=-=-=-=-=-=-=-=-=
%=-=-=-=-=-=-=-=-=-=-=-=-=-=-=-=-=-=-=-=-=-=-=-=-=-=-=-=-=-=-=-=-=-=-=-=-=-=-=-=
%   FRAME START   -=-=-=-=-=-=-=-=-=-=-=-=-=-=-=-=-=-=-=-=-=-=-=-=-=-=-=-=-=-=-=
\begin{frame}[c]{Optionality (IM$=$0/1)}

\textbf{The evolution of syntactic structures is characterized by optional stages} (IM$=$0/1).

\begin{table}[ht]
    \centering
    \begin{tabular}{|l|r|r|r|r|r|r|}
    \hline
     & Merge & Spell Out & Search & IM & Search\textsubscript{lex} & IM\textsubscript{lex} \\
    \hline
    Old Japanese ($=$Trevisan) & 1 & 0 & 1 & 1 & 0 & 0\\
    \hline
    \hl{Heian Japanese} & 1 & 0 & 1 & \hl{0/1} & 0 & 0 \\
    \hline
    Contemporary Japanese ($=$Italian) & 1 & 0 & 1 & 0 & 0 & 0 \\
    \hline
    \end{tabular}
    \caption{\label{tab:samp2}Parametrisations of FocP in the evolution of Japanese.}
    \end{table}

\end{frame}
%   FRAME END   --==-=-=-=-=-=-=-=-=-=-=-=-=-=-=-=-=-=-=-=-=-=-=-=-=-=-=-=-=-=-=
%=-=-=-=-=-=-=-=-=-=-=-=-=-=-=-=-=-=-=-=-=-=-=-=-=-=-=-=-=-=-=-=-=-=-=-=-=-=-=-=
%   FRAME START   -=-=-=-=-=-=-=-=-=-=-=-=-=-=-=-=-=-=-=-=-=-=-=-=-=-=-=-=-=-=-=
%\begin{frame}[c]{Different settings mean separate projections}



%\end{frame}
%   FRAME END   --==-=-=-=-=-=-=-=-=-=-=-=-=-=-=-=-=-=-=-=-=-=-=-=-=-=-=-=-=-=-=
%=-=-=-=-=-=-=-=-=-=-=-=-=-=-=-=-=-=-=-=-=-=-=-=-=-=-=-=-=-=-=-=-=-=-=-=-=-=-=-=
\end{document}