\documentclass[lesson_slides]{subfiles}
\usepackage{enumerate}
\usepackage{pifont} % for ding
\usepackage{gb4e} % leave last



\begin{document}
%%=-=-=-=-=-=-=-=-=-=-=-=-=-=-=-=-=-=-=-=-=-=-=-=-=-=-=-=-=-=-=-=-=-=-=-=-=-=-=-=
%   FRAME START   -=-=-=-=-=-=-=-=-=-=-=-=-=-=-=-=-=-=-=-=-=-=-=-=-=-=-=-=-=-=-=
\begin{frame}[c]{What's a parameter?}
    Rizzi (2017: 165): Parameter: 'an instruction for the triggering of a syntactic operation, expressed as a morphosyntactic feature
associated to a functional head'.
\end{frame}
%   FRAME END   --==-=-=-=-=-=-=-=-=-=-=-=-=-=-=-=-=-=-=-=-=-=-=-=-=-=-=-=-=-=-=
%=-=-=-=-=-=-=-=-=-=-=-=-=-=-=-=-=-=-=-=-=-=-=-=-=-=-=-=-=-=-=-=-=-=-=-=-=-=-=-=
%   FRAME START   -=-=-=-=-=-=-=-=-=-=-=-=-=-=-=-=-=-=-=-=-=-=-=-=-=-=-=-=-=-=-=
\begin{frame}[c]{Syntactic operations}
    \noindent\textbf{\textsc{syntactic operations}} 
    \begin{itemize}
        \item[\ding{227}] simple;
        \item[\ding{227}] highly learnable;
        \item[\ding{227}] restricted to an extremely reduced set for reasons of learnability.
    \end{itemize}	
\end{frame}
%   FRAME END   --==-=-=-=-=-=-=-=-=-=-=-=-=-=-=-=-=-=-=-=-=-=-=-=-=-=-=-=-=-=-=
%=-=-=-=-=-=-=-=-=-=-=-=-=-=-=-=-=-=-=-=-=-=-=-=-=-=-=-=-=-=-=-=-=-=-=-=-=-=-=-=
%   FRAME START   -=-=-=-=-=-=-=-=-=-=-=-=-=-=-=-=-=-=-=-=-=-=-=-=-=-=-=-=-=-=-=
\begin{frame}[c]{Syntactic operations (ii)}
    \noindent\textbf{\textsc{available syntactic operations}} 
        \begin{xlist}
            \ex Merge
            \ex Move
                \begin{xlist}
                    \ex Search:	Probe-goal relation at the phrasal level
                    \ex IM:	Internal merge of phrases; or:
                    \ex Search\textsubscript{lex}	Probe-goal relation at the head level
                    \ex IM\textsubscript{lex}	Internal merge of heads
                \end{xlist}
            \ex Spellout
        \end{xlist}
\end{frame}
%   FRAME END   --==-=-=-=-=-=-=-=-=-=-=-=-=-=-=-=-=-=-=-=-=-=-=-=-=-=-=-=-=-=-=
%=-=-=-=-=-=-=-=-=-=-=-=-=-=-=-=-=-=-=-=-=-=-=-=-=-=-=-=-=-=-=-=-=-=-=-=-=-=-=-=
%   FRAME START   -=-=-=-=-=-=-=-=-=-=-=-=-=-=-=-=-=-=-=-=-=-=-=-=-=-=-=-=-=-=-=
\begin{frame}[c]{Move}

    \noindent\textbf{\textsc{move}} 
    \begin{itemize}
        \item[\ding{227}] complex operation (à la \citealt{chomsky2001});
        \item[\ding{227}] involves either a head or a phrase;
        \item[\ding{227}] encompasses the establishment of a probe-goal search followed by (internal) merge of the goal. 
    \end{itemize}
        
\end{frame}
%   FRAME END   --==-=-=-=-=-=-=-=-=-=-=-=-=-=-=-=-=-=-=-=-=-=-=-=-=-=-=-=-=-=-=
%=-=-=-=-=-=-=-=-=-=-=-=-=-=-=-=-=-=-=-=-=-=-=-=-=-=-=-=-=-=-=-=-=-=-=-=-=-=-=-=
\begin{frame}[c]{Phrasal movement vs. Head movement}

    \noindent\textbf{\textsc{phrasal movement}} (\citealt{rizzi2017}: 171 (20))
        \begin{xlist}
            \ex A search feature at the phrasal level.
            \ex The corresponding internal merge feature at the phrasal level (IM) ('EPP feature').
        \end{xlist}

    \noindent\textbf{\textsc{head movement}} (\citealt{rizzi2017}: 171 (21)) 
        \begin{xlist}
            \ex A search feature at the lex level (Search\textsubscript{lex} Feature)
            \ex The corresponding internal merge feature, again at the lex level (IM\textsubscript{lex} Feature)
        \end{xlist}
        
\end{frame}

\end{document}